% !TEX TS-program = pdflatex
% !TEX encoding = UTF-8 Unicode

% This is a simple template for a LaTeX document using the "article" class.
% See "book", "report", "letter" for other types of document.

\documentclass[11pt]{article} % use larger type; default would be 10pt

\usepackage[utf8]{inputenc} % set input encoding (not needed with XeLaTeX)
\usepackage[ngerman]{babel}

%%% Examples of Article customizations
% These packages are optional, depending whether you want the features they provide.
% See the LaTeX Companion or other references for full information.

%%% PAGE DIMENSIONS
\usepackage{geometry} % to change the page dimensions
\geometry{a4paper} % or letterpaper (US) or a5paper or....
% \geometry{margin=2in} % for example, change the margins to 2 inches all round
% \geometry{landscape} % set up the page for landscape
%   read geometry.pdf for detailed page layout information

\usepackage{graphicx} % support the \includegraphics command and options

% \usepackage[parfill]{parskip} % Activate to begin paragraphs with an empty line rather than an indent

%%% PACKAGES
\usepackage{booktabs} % for much better looking tables
\usepackage{array} % for better arrays (eg matrices) in maths
\usepackage{paralist} % very flexible & customisable lists (eg. enumerate/itemize, etc.)
\usepackage{verbatim} % adds environment for commenting out blocks of text & for better verbatim
\usepackage{subfig} % make it possible to include more than one captioned figure/table in a single float
% These packages are all incorporated in the memoir class to one degree or another...

%%%%%%%%%%%%%%%%%%%%%%%%%%%%%%%%%%%%%%%%%%%%%%%%%
%Syntaxhighlighting:

\usepackage{listings}
\usepackage{etoolbox}
\usepackage{color}

\definecolor{base0}{RGB}{131,148,150}
\definecolor{base01}{RGB}{88,120,97}
\definecolor{base2}{RGB}{238,232,213}
\definecolor{sgreen}{RGB}{133,153,0}
\definecolor{sblue}{RGB}{28,118,200}
\definecolor{scyan}{RGB}{42,161,151}
\definecolor{smagenta}{RGB}{211,54,130}
\definecolor{hintergrund}{RGB}{240, 230, 230}


\newcommand\digitstyle{\color{smagenta}}
\newcommand\symbolstyle{\color{black}}
\makeatletter
\newcommand{\ProcessDigit}[1]
{%
  \ifnum\lst@mode=\lst@Pmode\relax%
   {\digitstyle #1}%
  \else
    #1%
  \fi
}
\makeatother


\lstdefinestyle{solarizedcsharp} {
  language=[Sharp]C,
  frame=lr,
  linewidth=160mm,
  breaklines=true,
  tabsize=2,
  numbers=left,
  numbersep=5pt,
  firstnumber=auto,
  numberstyle=\tiny\ttfamily\color{base01},
  rulecolor=\color{base2},
  basicstyle=\footnotesize\ttfamily,
  commentstyle=\color{base01},
  morecomment=[s][\color{base01}]{/*+}{*/},
  morecomment=[s][\color{base01}]{/*-}{*/},
  morekeywords={  abstract, event, new, struct,
                as, explicit, null, switch,
                base, extern, object, this,
                bool, false, operator, throw,
                break, finally, out, true,
                byte, fixed, override, try,
                case, float, params, typeof,
                catch, for, private, uint,
                char, foreach, protected, ulong,
                checked, goto, public, unchecked,
                class, if, readonly, unsafe,
                const, implicit, ref, ushort,
                continue, in, return, using,
                decimal, int, sbyte, virtual,
                default, interface, sealed, volatile,
                delegate, internal, short, void,
                do, is, sizeof, while,
                double, lock, stackalloc,
                else, long, static,
                enum, namespace, string, var},
  keywordstyle=\bfseries\color{sgreen},
  showstringspaces=false,
  stringstyle=\color{scyan},
  identifierstyle=\color{sblue},
  extendedchars=true,
  literate=
    {0}{{{\ProcessDigit{0}}}}1
    {1}{{{\ProcessDigit{1}}}}1
    {2}{{{\ProcessDigit{2}}}}1
    {3}{{{\ProcessDigit{3}}}}1
    {4}{{{\ProcessDigit{4}}}}1
    {5}{{{\ProcessDigit{5}}}}1
    {6}{{{\ProcessDigit{6}}}}1
    {7}{{{\ProcessDigit{7}}}}1
    {8}{{{\ProcessDigit{8}}}}1
    {9}{{{\ProcessDigit{9}}}}1
    {\}}{{\symbolstyle{\}}}}1
    {\{}{{\symbolstyle{\{}}}1
    {(}{{\symbolstyle{(}}}1
    {)}{{\symbolstyle{)}}}1
    {=}{{\symbolstyle{$=$}}}1
    {;}{{\symbolstyle{$;$}}}1
    {>}{{\symbolstyle{$>$}}}1
    {<}{{\symbolstyle{$<$}}}1
    {\%}{{\symbolstyle{$\%$}}}1
%deutsche sonderzeichen im Quelltext
    {ü}{{\"u}}1
    {Ü}{{\"U}}1
    {ö}{{\"o}}1
    {Ö}{{\"O}}1
    {ä}{{\"a}}1
    {Ä}{{\"A}}1
    {ß}{{\ss{}}}1,
}

\lstset{
escapechar=@,
style=solarizedcsharp, 
frame=l,
framesep=4.5mm,
fillcolor=\color{hintergrund}}

%%%%%%%%%%%%%%%%%%%%%%%%%%%%%%%%%%%%%%%%

%%% HEADERS & FOOTERS
\usepackage{fancyhdr} % This should be set AFTER setting up the page geometry
\pagestyle{fancy} % options: empty , plain , fancy
\renewcommand{\headrulewidth}{0pt} % customise the layout...
\lhead{}\chead{}\rhead{}
\lfoot{}\cfoot{\thepage}\rfoot{}

%%% SECTION TITLE APPEARANCE
\usepackage{sectsty}
\allsectionsfont{\sffamily\mdseries\upshape} % (See the fntguide.pdf for font help)
\usepackage{titlesec}
\newcommand{\sectionbreak}{\clearpage}
% (This matches ConTeXt defaults)

%%% ToC (table of contents) APPEARANCE
\usepackage[nottoc,notlof,notlot]{tocbibind} % Put the bibliography in the ToC
\setcounter{tocdepth}{2} % Show sections
\usepackage[titles,subfigure]{tocloft} % Alter the style of the Table of Contents
\renewcommand{\cftsecfont}{\rmfamily\mdseries\upshape}
\renewcommand{\cftsecpagefont}{\rmfamily\mdseries\upshape} % No bold!

%%% END Article customizations

%%% The "real" document content comes below...

\title{Virtual Reality für spielerisches Erkunden des HTWK Campus – Entwicklung einer Anwendung für die Oculus Quest mit Unity}
\author{Valentin Merz}
%\date{} % Activate to display a given date or no date (if empty),
         % otherwise the current date is printed 

% TODO: Profs erwähnen


\begin{document}
\pagenumbering{gobble}
\maketitle

\clearpage

\pagenumbering{roman}

\tableofcontents

\clearpage

\pagenumbering{arabic}


\section{Einleitung}

afd




\section{Hintergrund}

\subsection{Virtual Reality}

Begriffserklärung
Ältere Verwendung verglichen mit heute
Abgrenzung oder Übergang zu MR AR
Herausforderungen für Anwendungen in dem Medium

\subsection{Entwicklung in Unity}

Was ist eine GameEngine

Entity Component System

\subsection{Oculus Plugin}

Was für ein Design Pattern verwendet der Grabber?
Was für eins der Player Controller?
Was tut der OVR Manager?



\section{Konzeption}

%kommt etwas dazu, was nicht zu den subsections gehört?

\subsection{Nutzungsszenario} % Anwendungsszenario / Ziele / User Story

% Der folgende Satz ist aus der Konzeptionsemail kopiert
 Sich eine Übersicht an dem Campus zu verschaffen, sowie Informationen herauszusuchen kann eine ermüdende Aufgabe sein. Weil sie das ist gibt es dann auch Studierende, die nicht alle Institutionen kennen, die ihnen helfen könnten oder nicht alle Orte innerhalb des Campus kennen, die für sie interessant sein könnten. Indem die Tätigkeit des sich Informierens zu etwas Spielerischem wird ist sie, wenn auch nicht so schnell auch nicht erschöpfend und die Studierenden entdecken zumindest einige der für sie hilfreichen Informationen.


\subsection{Resultierende Anforderungen}

Da diese Anwendung für Menschen gemacht ist, die diese vermutlich nicht mehrfach nutzen ist es essentiell, dass es keine großen Hürden gibt mit, die erlernt werden müssen und dass Mechaniken, die nicht für die meisten Menschen offensichtlich sind innerhalb der Anwendung erklärt werden.

Die Gebäude der HTWK müssen klar erkennbar sein und die Information die man daraus ersehen kann muss korrekt sein.

Vollständigkeit ist nicht notwendig und auch nicht machbar.




\section{Technische Umsetzung}

\subsection{Zerlegen der Gebäude}

Eine Kernfunktion des Zerlegens sollte sein, dass die Gebäude aus den Einzelteilen einfach und richtig wieder zusammenbaubar sind. Dafür sollten die Einzelteile, wenn sie nah an ihren Verbindungsstücken sind wie Magnete passend aneinander kommen.\\

Für dieses aufeinander Zugehen der Teile werden immer Paare betrachtet, die im zusammengesetzten Modell eine sie verbindende Fläche haben. \\

Überprüfen distanz collider vs know it  +  Identifizieren durch in der Szene übergeben vs tags \\

Abprallen und das verhindern dessen\\

Greifen und Hochheben von ganzen Gebäuden \\

Joints und daraus resultierendes Zittern vs: \\

Versuch des abänderns von Oculus scripts um mehrere Objekte hochzuheben 

\subsection{offhandgrab / zerlegen mit beiden Händen}

Wenn mit der einen hand gegriffen wird und mit der anderen verbundene Teile gegriffen werden was soll passieren?

Ich sehe drei Design Möglichkeiten, von denen ich nicht sicher bin, welche die Beste ist:
1. Teilen in der Mitte
2. Teilen neben der neuen Hand
3. alles zwischen den Händen fällt

Es soll sich natürlich anfühlen und vorhersehbar sein, nachdem man damit schon mal interagiert hat.
Wenn es interessant sein kann, oder einen Spaß Moment des Zerstörens haben kann wäre das gut.
Option 3 erscheint am riskantesten, aber als hätte es viel Potential. 
Ein aufteilen in der Mitte klingt erst mal sicher, aber vielleicht fühlt es sich sehr ungenau an.
Direkt neben der neuen Hand klingt exakter, aber vielleicht bei der ersten Erfahrung unerwarteter.

Wie ist das machbar?

das Gegriffene Gebäude als Graph zu betrachten liegt nahe, da es Einzelteile sind, bei denen immer Verbindungen zwischen zwei Teilen bestehen.
Wenn dieser Graph ein Pfad ist gibt es keine Komplikationen ihn aufzuteilen, aber wenn er stärker vernetzt ist erscheint es nicht mehr so offensichtlich, was alles an dem neu gegriffenen hängt.
Vielleicht ist die Lösung nur die Teile, die nur durch den gegriffenen verbunden sind als an ihm hängend zu betrachten.
Derzeit erscheint es mir aber sinnvoller räumliche Nähe mit einfließen zu lassen, oder Nähe im Graphen einfließen zu lassen: Falls ein Teil mit keinem Kürzeren Weg mit der Ursprungshand verbunden ist, als der Weg über das neu gegriffene Teile wäre dieses als am neuen Hängend zu betrachten.
Intuitiv erscheint es mir sinnvoll den Gleichheitsfall in diesem Vergleich der neuen Hand zuzuordnen, aber sicher bin ich damit nicht.

Falls man räumliche Nähe heranzieht wäre es wichtig tatsächlich die räumliche Mitte der Teile zu vergleichen und nicht transforms, die an ihren Rändern liegen. Auch ist dabei dann die Frage, ob man die Verbindungsstruktur immer noch untersuchen muss, weil es in manchen Fällen sonst zu Problemen könnte, oder ob das dann als alleiniges Kriterium reichen wird.

Derzeit haben die Einzelteile kein Konzept ihres ganzen und speichern nur ihre direkt verbundenen, worüber sich die das gesamte Netz abfragen lässt. vielleicht wäre es nützlich einen Wert zu speichern, der die Distanz vom Einzelteil zu der Hand, die es direkt oder indirekt greift, festhält, aber wenn man diesen nur in der Situation benötigt, in der zwei Hände greifen würde dies ja nur ein wenig aufwand sparen im Vergleich dazu ihn jedes mal zu bestimmen.
Eine kürze Weg Suche klingt aber recht aufwändig, wobei die Graphen ja meist recht klein sein werden, also ist es vermutlich nur aufwändig zu implementieren, falls überhaupt.



\subsection{Nutzbarkeit der exisstierenden 3D Modelle}

asdf

\subsection{Erzeugen eines aufschlussreichen Innenlebens}

Das Dezernat Technik hat mir Gebäudepläne für den Gutenbergbau zur Verfügung gestellt.

\subsection{Performance}

Collider vs Distanzüberprüfen in Update

Fremde Gebäudemodelle




\section{Ausblick auf mögliche Anwendung und Erweiterung}

ljkhg




\section {Fazit}

adf




\section {Anhang}

\subsection{Glossar}

%Zu erklärende Begriffe:
Rigidbody
Joint
Collider
Oculus
Quest
Unity
subclass?

\section{Notizen}

Ich habe ein Dictionary statt eines Hashtables verwendet, weil ich nicht gemischte Typen verwenden möchte. 
https://docs.microsoft.com/en-us/dotnet/api/system.collections.generic.dictionary-2?view=net-5.0


Typecasting ist vermutlich verwendet, wenn ich ein GreifbaresEinzelteil spezifisch und nicht wie ein OVRGrabbable behandeln möchte. 
https://docs.microsoft.com/de-de/dotnet/csharp/programming-guide/types/casting-and-type-conversions
damit konnte eine leere Funktion gespart werden:

\begin{lstlisting}
// diese Methode erscheint notwendig um die der subklasse Greifbares Einzelteil effektiv nutzen 
// zu können. Da die Baseclass sich nicht mit anderen Stücken verbindet gibt sie eine leere Menge
// zurück. TODO: etwas eleganteres finden
public virtual HashSet<OVRGrabbable> alleVerbundenen(HashSet<OVRGrabbable> schonGefundene) {
	return schonGefundene;
}
\end{lstlisting}

stattdessen in dem Greifer: 

\begin{lstlisting}
if (m_grabbedObj is GreifbaresEinzelteil) {
	GreifbaresEinzelteil gegriffenesEinzelteil = (GreifbaresEinzelteil)m_grabbedObj;
	foreach (var k in gegriffenesEinzelteil.alleVerbundenen(new HashSet<OVRGrabbable>())) {
		k.grabbedRigidbody.MovePosition(grabbablePosition);
		k.grabbedRigidbody.MoveRotation(grabbableRotation);
	}
}
\end{lstlisting}

Tuples statt Array, weil verschiedene Typen.


\end{document}
