% !TEX TS-program = pdflatex
% !TEX encoding = UTF-8 Unicode

% This is a simple template for a LaTeX document using the "article" class.
% See "book", "report", "letter" for other types of document.

\documentclass[11pt]{article} % use larger type; default would be 10pt

\usepackage[utf8]{inputenc} % set input encoding (not needed with XeLaTeX)
\usepackage[ngerman]{babel}

%%% Examples of Article customizations
% These packages are optional, depending whether you want the features they provide.
% See the LaTeX Companion or other references for full information.

%%% PAGE DIMENSIONS
\usepackage{geometry} % to change the page dimensions
\geometry{a4paper} % or letterpaper (US) or a5paper or....
% \geometry{margin=2in} % for example, change the margins to 2 inches all round
% \geometry{landscape} % set up the page for landscape
%   read geometry.pdf for detailed page layout information

\usepackage{graphicx} % support the \includegraphics command and options

%\usepackage[parfill]{parskip} % Activate to begin paragraphs with an empty line rather than an indent
\setlength{\parskip}{1em}

%%% PACKAGES
\usepackage{booktabs} % for much better looking tables
\usepackage{array} % for better arrays (eg matrices) in maths
\usepackage{paralist} % very flexible & customisable lists (eg. enumerate/itemize, etc.)
\usepackage{verbatim} % adds environment for commenting out blocks of text & for better verbatim
\usepackage{subfig} % make it possible to include more than one captioned figure/table in a single float
% These packages are all incorporated in the memoir class to one degree or another...

%%%%%%%%%%%%%%%%%%%%%%%%%%%%%%%%%%%%%%%%%%%%%%%%%
%Syntaxhighlighting:

\usepackage{listings}
\usepackage{etoolbox}
\usepackage{color}

\definecolor{base0}{RGB}{131,148,150}
\definecolor{base01}{RGB}{88,120,97}
\definecolor{base2}{RGB}{238,232,213}
\definecolor{sgreen}{RGB}{133,153,0}
\definecolor{sblue}{RGB}{28,118,200}
\definecolor{scyan}{RGB}{42,161,151}
\definecolor{smagenta}{RGB}{211,54,130}
\definecolor{hintergrund}{RGB}{240, 230, 230}


\newcommand\digitstyle{\color{smagenta}}
\newcommand\symbolstyle{\color{black}}
\makeatletter
\newcommand{\ProcessDigit}[1]
{%
  \ifnum\lst@mode=\lst@Pmode\relax%
   {\digitstyle #1}%
  \else
    #1%
  \fi
}
\makeatother


\lstdefinestyle{solarizedcsharp} {
  language=[Sharp]C,
  frame=lr,
  linewidth=160mm,
  breaklines=true,
  tabsize=2,
  numbers=left,
  numbersep=5pt,
  firstnumber=auto,
  numberstyle=\tiny\ttfamily\color{base01},
  rulecolor=\color{base2},
  basicstyle=\footnotesize\ttfamily,
  commentstyle=\color{base01},
  morecomment=[s][\color{base01}]{/*+}{*/},
  morecomment=[s][\color{base01}]{/*-}{*/},
  morekeywords={  abstract, event, new, struct,
                as, explicit, null, switch,
                base, extern, object, this,
                bool, false, operator, throw,
                break, finally, out, true,
                byte, fixed, override, try,
                case, float, params, typeof,
                catch, for, private, uint,
                char, foreach, protected, ulong,
                checked, goto, public, unchecked,
                class, if, readonly, unsafe,
                const, implicit, ref, ushort,
                continue, in, return, using,
                decimal, int, sbyte, virtual,
                default, interface, sealed, volatile,
                delegate, internal, short, void,
                do, is, sizeof, while,
                double, lock, stackalloc,
                else, long, static,
                enum, namespace, string, var},
  keywordstyle=\bfseries\color{sgreen},
  showstringspaces=false,
  stringstyle=\color{scyan},
  identifierstyle=\color{sblue},
  extendedchars=true,
  literate=
    {0}{{{\ProcessDigit{0}}}}1
    {1}{{{\ProcessDigit{1}}}}1
    {2}{{{\ProcessDigit{2}}}}1
    {3}{{{\ProcessDigit{3}}}}1
    {4}{{{\ProcessDigit{4}}}}1
    {5}{{{\ProcessDigit{5}}}}1
    {6}{{{\ProcessDigit{6}}}}1
    {7}{{{\ProcessDigit{7}}}}1
    {8}{{{\ProcessDigit{8}}}}1
    {9}{{{\ProcessDigit{9}}}}1
    {\}}{{\symbolstyle{\}}}}1
    {\{}{{\symbolstyle{\{}}}1
    {(}{{\symbolstyle{(}}}1
    {)}{{\symbolstyle{)}}}1
    {=}{{\symbolstyle{$=$}}}1
    {;}{{\symbolstyle{$;$}}}1
    {>}{{\symbolstyle{$>$}}}1
    {<}{{\symbolstyle{$<$}}}1
    {\%}{{\symbolstyle{$\%$}}}1
%deutsche sonderzeichen im Quelltext
    {ü}{{\"u}}1
    {Ü}{{\"U}}1
    {ö}{{\"o}}1
    {Ö}{{\"O}}1
    {ä}{{\"a}}1
    {Ä}{{\"A}}1
    {ß}{{\ss{}}}1,
}

\lstset{
escapechar=@,
style=solarizedcsharp, 
frame=l,
framesep=4.5mm,
fillcolor=\color{hintergrund}}

%%%%%%%%%%%%%%%%%%%%%%%%%%%%%%%%%%%%%%%%

%%% HEADERS & FOOTERS
\usepackage{fancyhdr} % This should be set AFTER setting up the page geometry
\pagestyle{fancy} % options: empty , plain , fancy
\renewcommand{\headrulewidth}{0pt} % customise the layout...
\lhead{}\chead{}\rhead{}
\lfoot{}\cfoot{\thepage}\rfoot{}

%%% SECTION TITLE APPEARANCE
\usepackage{sectsty}
\allsectionsfont{\sffamily\mdseries\upshape} % (See the fntguide.pdf for font help)
\usepackage{titlesec}
\newcommand{\sectionbreak}{\clearpage}
% (This matches ConTeXt defaults)

%%% ToC (table of contents) APPEARANCE
\usepackage[nottoc,notlof,notlot]{tocbibind} % Put the bibliography in the ToC
\setcounter{tocdepth}{2} % Show sections
\usepackage[titles,subfigure]{tocloft} % Alter the style of the Table of Contents
\renewcommand{\cftsecfont}{\rmfamily\mdseries\upshape}
\renewcommand{\cftsecpagefont}{\rmfamily\mdseries\upshape} % No bold!

%%% END Article customizations

%%% The "real" document content comes below...

\title{Virtual Reality für spielerisches Erkunden des HTWK Campus – Entwicklung einer Anwendung für die Oculus Quest mit Unity}
\author{Valentin Merz}
%\date{} % Activate to display a given date or no date (if empty),
         % otherwise the current date is printed 

% TODO: Profs erwähnen


\begin{document}
\pagenumbering{gobble}
\maketitle

\clearpage

\pagenumbering{roman}

\tableofcontents

\clearpage

\pagenumbering{arabic}


\section{Einleitung}

\subsection{Geschlechtsneutrale Sprache}

Statement zu generischem Femininum

Oder darf ich Nutzis Spielis etc benutzen statt Nutzer_innen oder Spieler*innen und mensch statt man?

Verwendung von Ich?

\section{Hintergrund}

\subsection{Virtual Reality}

Begriffserklärung
Ältere Verwendung verglichen mit heute??
Abgrenzung oder Übergang zu MR AR??
Herausforderungen für Anwendungen in dem Medium

\subsection{Entwicklung in Unity}

Was ist eine GameEngine

Entity Component System??

\subsection{Oculus Plugin}

Was für ein Design Pattern verwendet der Grabber?
Was für eins der Player Controller?
Was tut der OVR Manager?



\section{Konzeption}

%kommt etwas dazu, was nicht zu den subsections gehört?

\subsection{Nutzungsszenario} % Anwendungsszenario / Ziele / User Story

% Der folgende Satz ist aus der Konzeptionsemail kopiert
 Sich eine Übersicht an dem Campus zu verschaffen, sowie Informationen herauszusuchen kann eine ermüdende Aufgabe sein. Weil sie das ist gibt es dann auch Studierende, die nicht alle Institutionen kennen, die ihnen helfen könnten oder nicht alle Orte innerhalb des Campus kennen, die für sie interessant sein könnten. Indem die Tätigkeit des sich Informierens zu etwas Spielerischem wird ist sie, wenn auch nicht so schnell auch nicht erschöpfend und die Studierenden entdecken zumindest einige der für sie hilfreichen Informationen.


\subsection{Resultierende Anforderungen}

Da diese Anwendung für Menschen gemacht ist, die diese vermutlich nicht mehrfach nutzen ist es essentiell, dass es keine großen Hürden gibt mit, die erlernt werden müssen und dass Mechaniken, die nicht für die meisten Menschen offensichtlich sind innerhalb der Anwendung erklärt werden.

Die Gebäude der HTWK müssen klar erkennbar sein und die Information die man daraus ersehen kann muss korrekt sein.

Vollständigkeit ist nicht notwendig und auch nicht machbar.

\subsection{Vergleichsanwendungen}

\subsubsection{Google Earth VR}

\subsubsection{Museum}

\subsection{Vision}

%Spielzeugmodelle: 
Das Ziel die Gebäude kennen zu lernen möchte ich verfolgen indem man die Gebäude als Spielzeug in der Welt hat. Anstatt also durch die Straßen zu gehen, so wie man das auch in der echten Welt, oder Google Street view auch könnte hat man eine Form von Überblick, die ein bisschen aus der Vogelperspektive ist, aber auch ermöglicht sich Details anzuschauen, da man Gebäude auch hochheben und genauer betrachten kann.

%Bewegung gelöst durch kleine Gebäude
Wenn die Gebäude klein dargestellt werden hat das zudem den Vorteil, dass man nicht eine Welt durchqueren muss, die deutlich größer als der für den Spieler in der Echten Welt zur Verfügung stehende Bereich ist. Somit kann es möglich sein, dass alle Bewegung des Charakters im Spiel auch von dem Menschen, der die VR Brille nutzt ausgeführt werden, was Immersion unterstützt. [citation needed]

%Überblick 
Die Perspektive, die man auf den Campus erhält, wenn dieser in Spielzeuggröße dargestellt ist eine nicht häufig vorkommende: Ganz von oben sieht man ihn auf Karten und in Realgröße in der Realität. Eine Abbildungsgröße dazwischen könnte einen Überblick bieten und gleichzeitig ein gutes visuelles Wiedererkennen der Gebäude ermöglichen. Da es ja auch möglich ist die Gebäude zu nehmen kann man zudem auch Details anschauen, soweit die Modelle das zulassen.

%Öffnen %hatte ich hier als Stichpunkt stehen, aber weiß nicht, was das anderes als Auseinandernehmen oder vielleicht Reinschauen sein soll

%Reinschauen: 
In die Gebäude hineinzuschauen fügt viel wichtigen Einblick hinzu, da gerade ein Verständnis der Gänge hilft zu verstehen wie man sich durch die Hochschule bewegen kann. Außerdem kann man erkennen wo wichtige Orte sind und wie die Raumnummern verteilt sind. 
%Auseinandernehmen: 
Dem Nutzer die Möglichkeit zu geben die Spielzeuggebäude in mehrere Stücke zu zerteilen kann dieses Hineinschauen erlauben. Wenn dabei die Stücke in die die Gebäude zerlegt werden nicht durch die Physik der Anwendung oder ähnliches, sondern durch im Vorhinein festgelegte Einzelteile bestimmt werden gibt es hier auch die Möglichkeit bestimmte Eigenheiten oder besonders wichtige Orte hervorgehoben zu präsentieren.

%Spieltrieb: 
Gegenstände zerlegen und wieder zusammenbauen zu können spricht auch einen Spieltrieb an [citation needed], der ja einer der großen Gründe dafür ist, dass diese Anwendung in diesem Medium das Potential eines Mehrwerts bietet.
%Kinderzimmer:
Mit Modellspielzeug zu spielen passt sehr natürlich in das Setting eines Kinderzimmers. Das, sowie den Spieler in Kindergröße schlüpfen zu lassen, soll ihm helfen aus seiner gewohnten Umgebung herauszugehen und sich voll kindlicher Neugier in das Erkunden zu stürzen.

\subsection{nice-to-have features}

%Teppich:
Die Gebäude und der Spieler können auf einem Spielteppich platziert sein, welcher die Anordnung der Gebäude sowie ihre Nummern aufgedruckt hat, sodass wenn man diese Hochhebt man sieht wo man sie weggenommen hat und die Einordnung einfach fällt. Hierauf könnten auch die nächsten Haltestellen oder andere nützliche Umgebungsinformationen zu finden sein. 

%Gegenstände in den Räumen der zerlegbaren Gebäude:
In manchen der Räume können Gegenstände liegen ähnlich zu einem Puppenhaus, sodass diese Räume hervorgehoben werden und ihre Funktion deutlich wird. Ein Beispiel wäre, dass in dem Entsprechenden Gang ein Prüfungsplan hängt, oder in einem PC Pool ein Computer steht, in den WCs eine Toilette oder in den PC-Pools in denen die Drucker nutzbar sind ein Drucker.

%Interaktion
Bei manchen davon erscheint es sinnvoll, dass sie fest sind und nur durch ihre Existenz signalisieren, was es dort gibt. An anderen Stellen kann es nützlich sein dass sie herausnehmbar sind und beim greifen etwas tun, wie Töne ausgeben, oder weitere Information einblenden. Zum Beispiel könnte ein Computer angehen und zeigen was für ein Betriebssystem in diesem Raum installiert ist.
%TODO: Mehr Interaktions und Spielzeugkistenmöglichkeiten und Aufgaben

%Spielzeugkiste mit was genau
Für Information, die zwar mit einem Gegenstand gut zu symbolisieren oder interessant verwirklicht werden kann, aber keinen klaren räumlichen Bezug ist auch eine Spielzeugkiste, die am Rand steht eine Möglichkeit. 

%Poster
Genauso würde es in das Setting passen Poster an die Wände zu hängen um dort etwas allgemeines anzuzeigen wie zum Beispiel erste Bedienschritte. Hier ist es auch möglich dass diese sich ändern und nachdem der jeweilige Bedienschritt ausgeführt wurde auf eine neue Möglichkeit mit der Umgebung zu interagieren hinweist oder eine Übersicht darstellt. Auch Poster zu weiteren Informationen über die Hochschule, denen man große Bedeutung beimisst und nicht an anderer Stelle einbinden kann sind möglich.

%Aufgaben oder Ziele für den Spieler
Erkunden basiert auf selbst gesetzten Zielen ist toll, [citation needed] aber nach einem ersten Umschauen Anregungen zu erhalten nach was man suchen könnte helfen falls bei Spielern schnell eine Ziellosigkeit eintritt.
Wenn manche der in den Räumen verteilten Gegenständen den Spieler auffordern etwas auszuprobieren oder zu finden ohne das strikt vorzuschreiben könnte die Selbstbestimmung unverletzt bleiben und trotzdem eine Vorstellung von Arten sich mit der präsentierten Information auseinanderzusetzen angeboten werden. Ein Beispiel hier wäre ein Rollstuhl, der einen bei Kontakt auffordert ihn entlang barrierefreier Pfade an einen bestimmten Ort zu bewegen oder alternativ sich nur entlang dieser Wege bewegen kann und aufruft auszuprobieren, wie man ihn bewegen kann. 

%Der Boden ist Lava %Playspace erklärt
In einer VR Anwendung ist es wichtig, dass die Nutzerin weiß in welchem Raum sie sich bewegen kann und wann sie in Gegenstände oder Wände in der echten Umgebung laufen würde. Hierfür bietet Oculus eine Lösung mit einem blauen Gitter, dass die zuvor abgesteckten Grenzen des Playspace %Ist das so klar genug?
darstellt und erscheint, wenn man diesen nah kommt. Man könnte diese Lösung ersetzen indem der Teppich die Form des möglichen Raumes hat und der Boden, wenn man ihm nahekommt, mit einer Lavatextur aufleuchtet und vielleicht noch eine Stimme sagt ``Der Boden ist Lava'' erscheint. Dies würde das kindliche Spielszenario verstärken, Thematisch passen und die Kommunikation des Playspace aufrechterhalten, auch wenn die Spielerin den Grenzen nicht nahe ist. Vielleicht fühlen sich aber auch manche Nutzerinnen damit unwohl so stark in eine kindliche Rolle zu schlüpfen. %Spekulation?

%Wiederaufbauhilfen
Es könnte auch gut sein der Spielerin eine einfach Möglichkeit anzubieten den Startzustand wiederherzustellen. Chaos verursachen macht ja meist mehr Spaß als wieder aufzuräumen. Eine mit ``Hilf mir aufräumen'' Beschriftete Box mit großem roten Knopf könnte das klar kommunizieren.

\subsection{Evaluation anhand der Literatur}

\subsubsection{On the Design of Virtual Reali... in Engineering} %TODO bib einbinden und anständig zitieren!!!

Hier werden zwei notwendige Kriterien aufgeführt damit die Entwicklung sinnvoll ist:
1. Die Ressource muss zu einem besseren Lehr-Lern-Prozess führen
2. Es muss die Kosten wert sein 
In beiden Aspekten bin ich mir bei dieser Anwendung nicht sicher.

\section{Technische Umsetzung}

\subsection{Zerlegen der Gebäude}

Eine Kernfunktion des Zerlegens sollte sein, dass die Gebäude aus den Einzelteilen einfach und richtig wieder zusammenbaubar sind. Dafür sollten die Einzelteile, wenn sie nah an ihren Verbindungsstücken sind wie Magnete passend aneinander kommen.

Für dieses aufeinander Zugehen der Teile werden immer Paare betrachtet, die im zusammengesetzten Modell eine sie verbindende Fläche haben. 

Überprüfen distanz collider vs know it  +  Identifizieren durch in der Szene übergeben vs tags 

Abprallen und das verhindern dessen

Mehrere Einzelteile, die schon zusammen sind sollen an einer Stelle genommen und bewegt werden können und sich dabei wie ein Ganzes verhalten.
Um das zu erreichen versuchte ich als erstes fixed joints zu verwenden, über die das Unity Manual folgendes schreibt: 
''Restricts the movement of a rigid body to follow the movement of the rigid body it is attached to. This is useful when you need rigid bodies that easily break apart from each other, or you want to connect the movement of two rigid bodies without parenting in a Transform hierarchy.'' 

https://docs.unity3d.com/Manual/Joints.html

Anhand dieser Beschreibung erschienen sie wie eine hervorragende Lösung, bei der Verwendung erschien dann allerdings ein Verhalten, dass sie als alleinige Lösung für unbrauchbar erscheinen ließ: Wenn man ein Ende greift und bewegt erscheint nicht der gesamte Block als ganzes bewegt zu werden, sondern alle mit dem primär gegriffenen Einzelteil verbundenen Stücke ruckelten hinterher, was unangenehm anzuschauen war und nicht dem gewünschten Verhalten entsprach.
Sie konnten dabei auch auseinanderbrechen und in der Luft hängenbleiben.
Das Glossar erwähnt tatsächlich etwas derartiges: \newline
''Fixed Joint: A joint type that is completely constrained, allowing two objects to be held together. Implemented as a spring so some motion may still occur.'' \newline
% https://docs.unity3d.com/Manual/Glossary.html#FixedJoint

Mein nächster Lösungsansatz war die Greif Funktion aus dem Oculus Plugin für meinen speziellen Fall abzuwandeln. Hierfür schrieb ich die Unterklassen GreifbaresEinzelteil als Unterklasse von OVRGrabbable und MehrfachGreifer von OVRGrabber.
Diese sehen vor, dass mehr als nur ein Objekt von einer Hand hochgehoben werden kann und alle mit dem primär gegriffenen Einzelteil verbundenen Teile von der Hand bewegt werden. Dadurch muss dies nicht mehr von Unitys Physics Engine über Fixed Joints übernommen werden.
Es sorgte dafür, dass die Bewegung ordentlich abläuft und ich nehme auch an, dass es etwas performanter sein dürfte.

% so ne Spekulation gehört vermutlich nicht hier rein. möchte ich das testen? ist das relevant? Aber ich glaube dass es war sein sollte

Funktionen dieser zwei Klassen ausführen

Weiterhin bestehende Probleme: 
Teile rütteln aneinander und 
es ist schwer sie richtig zusammen zu setzen

\subsection{offhandgrab / zerlegen mit beiden Händen}

Wenn mit der einen hand gegriffen wird und mit der anderen verbundene Teile gegriffen werden was soll passieren?

Ich sehe drei Design Möglichkeiten, von denen ich nicht sicher bin, welche die Beste ist:
1. Teilen in der Mitte
2. Teilen neben der neuen Hand
3. alles zwischen den Händen fällt

Es soll sich natürlich anfühlen und vorhersehbar sein, nachdem man damit schon mal interagiert hat.
Wenn es interessant sein kann, oder einen Spaß Moment des Zerstörens haben kann wäre das gut.
Option 3 erscheint am riskantesten, aber als hätte es viel Potential. 
Ein aufteilen in der Mitte klingt erst mal sicher, aber vielleicht fühlt es sich sehr ungenau an.
Direkt neben der neuen Hand klingt exakter, aber vielleicht bei der ersten Erfahrung unerwarteter.

Wie ist das machbar?

das Gegriffene Gebäude als Graph zu betrachten liegt nahe, da es Einzelteile sind, bei denen immer Verbindungen zwischen zwei Teilen bestehen.
Wenn dieser Graph ein Pfad ist gibt es keine Komplikationen ihn aufzuteilen, aber wenn er stärker vernetzt ist erscheint es nicht mehr so offensichtlich, was alles an dem neu gegriffenen hängt.
Vielleicht ist die Lösung nur die Teile, die nur durch den gegriffenen verbunden sind als an ihm hängend zu betrachten.
Derzeit erscheint es mir aber sinnvoller räumliche Nähe mit einfließen zu lassen, oder Nähe im Graphen einfließen zu lassen: Falls ein Teil mit keinem Kürzeren Weg mit der Ursprungshand verbunden ist, als der Weg über das neu gegriffene Teile wäre dieses als am neuen Hängend zu betrachten.
Intuitiv erscheint es mir sinnvoll den Gleichheitsfall in diesem Vergleich der neuen Hand zuzuordnen, aber sicher bin ich damit nicht.

Falls man räumliche Nähe heranzieht wäre es wichtig tatsächlich die räumliche Mitte der Teile zu vergleichen und nicht transforms, die an ihren Rändern liegen. Auch ist dabei dann die Frage, ob man die Verbindungsstruktur immer noch untersuchen muss, weil es in manchen Fällen sonst zu Problemen könnte, oder ob das dann als alleiniges Kriterium reichen wird.

Derzeit haben die Einzelteile kein Konzept ihres ganzen und speichern nur ihre direkt verbundenen, worüber sich die das gesamte Netz abfragen lässt. vielleicht wäre es nützlich einen Wert zu speichern, der die Distanz vom Einzelteil zu der Hand, die es direkt oder indirekt greift, festhält, aber wenn man diesen nur in der Situation benötigt, in der zwei Hände greifen würde dies ja nur ein wenig aufwand sparen im Vergleich dazu ihn jedes mal zu bestimmen.
Eine kürze Weg Suche klingt aber recht aufwändig, wobei die Graphen ja meist recht klein sein werden, also ist es vermutlich nur aufwändig zu implementieren, falls überhaupt.



\subsection{Nutzbarkeit der exisstierenden 3D Modelle}

asdf

\subsection{Erzeugen eines aufschlussreichen Innenlebens}

Das Dezernat Technik hat mir Gebäudepläne für den Gutenbergbau zur Verfügung gestellt.

\subsection{Performance}

Collider vs Distanzüberprüfen in Update

Fremde Gebäudemodelle




\section{Ausblick auf mögliche Anwendung und Erweiterung}

ljkhg




\section {Fazit}

adf




\section {Anhang}

\subsection{Glossar}

%Zu erklärende Begriffe:
Rigidbody
Joint
Collider
Oculus
Quest
Unity
subclass?

\section{Notizen}

Ich habe ein Dictionary statt eines Hashtables verwendet, weil ich nicht gemischte Typen verwenden möchte. 
https://docs.microsoft.com/en-us/dotnet/api/system.collections.generic.dictionary-2?view=net-5.0


Typecasting ist vermutlich verwendet, wenn ich ein GreifbaresEinzelteil spezifisch und nicht wie ein OVRGrabbable behandeln möchte. 
https://docs.microsoft.com/de-de/dotnet/csharp/programming-guide/types/casting-and-type-conversions
damit konnte eine leere Funktion gespart werden:

\begin{lstlisting}
// diese Methode erscheint notwendig um die der subklasse Greifbares Einzelteil effektiv nutzen 
// zu können. Da die Baseclass sich nicht mit anderen Stücken verbindet gibt sie eine leere Menge
// zurück. TODO: etwas eleganteres finden
public virtual HashSet<OVRGrabbable> alleVerbundenen(HashSet<OVRGrabbable> schonGefundene) {
	return schonGefundene;
}
\end{lstlisting}

stattdessen in dem Greifer: 

\begin{lstlisting}
if (m_grabbedObj is GreifbaresEinzelteil) {
	GreifbaresEinzelteil gegriffenesEinzelteil = (GreifbaresEinzelteil)m_grabbedObj;
	foreach (var k in gegriffenesEinzelteil.alleVerbundenen(new HashSet<OVRGrabbable>())) {
		k.grabbedRigidbody.MovePosition(grabbablePosition);
		k.grabbedRigidbody.MoveRotation(grabbableRotation);
	}
}
\end{lstlisting}

Tuples statt Array, weil verschiedene Typen.


\end{document}
